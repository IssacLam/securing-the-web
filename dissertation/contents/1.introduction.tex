% What is the problem?
% Why is it interesting and important?
% Why is it hard? (E.g., why do naive approaches fail?)
% Why hasn't it been solved before? (Or, what's wrong with previous proposed solutions? How does mine differ?)
% What are the key components of my approach and results? Also include any specific limitations.

\chapter{Introduction}

\section{Motivation} \label{sec:heartbleed bug}
%% add affect & how serious it is %%
Failure of ensuring the safety of network infrastructure components could cause confidential information leakage and fatal damages to enterprises. In April 2014, a simple careless mistake caused a security nightmare, namely the Heartbleed Bug\footnote{The Heartbleed Bug: http://heartbleed.com/}, to the Internet disclosed by both Google's Security team and Codenomicon separately. It is a serious vulnerability in the extensively used Transport Layer Security (TLS) protocol implementation, OpenSSL cryptographic library\footnote{The OpenSSL Github repository: https://github.com/openssl/openssl}. This vulnerability enabled attackers to steal sensitive information from HTTPS services and even impersonate services or users. In this case, about half a million (17.5\%) of trusted HTTPS websites were influenced, which included banking services, email services, cloud storage, and any web services using the vulnerable versions of OpenSSL, according to the Secure Sockets Layer (SSL)\footnote{Secure Sockets Layer: the predecessor of TLS, often be called TLS/SSL} survey conducted by the Netcraft \cite{9_heartbleed}.

%% Explain how it fails and can be avoid by using software verification %%
This security weakness occurred in the TLS "heartbeat" extension, which is a keep-alive feature designed to ensure the connection by sending an arbitrary payload to, and receiving the exact same copy of it from the other end. The weakness was due to the fact that the implementation mistakenly relied on non-sanitised user input and took it as the length parameter of \code{memcpy()} without any bounds check \cite{6_heartbleed_bug}. Such a missing of a proper input validation makes the reading of the payload beyond the end of the buffer. Hence, attackers were able to extract the memory contents, such as identifications of service providers, secret keys for encrypting communications and users login information under the protection of vulnerable versions of OpenSSL in the server. In other words, they can steal data from communications directly and impersonate the services and users. However, this serious security vulnerability caused by buffer overruns can be easily detected using formal software verification techniques, for example, Bounded Model Checking, as demonstrated by the study of M. Voelter et al. \cite{Voelter:2015:TIS:2846696.2846698}. Therefore, not only the approach itself must be formally proven secure, but also the corresponding implementations are necessary to be verified.

Since data integrity and privacy of Internet communications for the online applications, especially for those in financial usage, and virtual private networks (VPNs) are provided by TLS/SSL. Therefore, ensuring the safety of the TLS/SSL implementation is particular importance so as to secure the web. This project is motivated by the necessity of ensuring the safety and correctness of such critical software components and the success of software verification. We aim at verifying memory safety of the Amazon's s2n due to its characteristics of light-weight implementation of the TLS protocol using a bounded model checking tool called CBMC.


% \begin{figure}
%     \centering
%     \begin{minted}[fontsize=\footnotesize, linenos, frame=single]{c}
%     struct {
%         uint16 payload_length;
%         opaque payload[HeartbeatMessage.payload_length];
%         ...
%     } HearbeatMessage;
%     \end{minted}
%     \caption{The Heartbeat vulnerable code.}
%     \label{fig:thvc}
% \end{figure}



% Towards Improving Software Security using Language Engineering and mbeddr C.
% Many of the software security weakness originate from careless or wrong use of programming language.
% Software security refer to the security properties of a software;s implementation.

% The study further conclude that most of the software security weakness caused by careless or wrong use of programming languages, such as the Heartbleed bug, are possible to be prevented under a formal software verification.


% The proof that the program satisfies the specification follows from the fact that all trajectories are in the abstraction and this abstraction does not intersect the bad states.


% Internet surfing has become one of the important parts of life for most of the human being. No matter what we use it for, such as reading the daily news, watching our favourite sports or shopping online, our personal or sensitive information is always involved in the data transmission. Moreover, we will also assume that those data are safely handled by the web application provider. However, there are many web infrastructure components, such as establishing the connection, data encryption and security involved in the data transmissions between the actual application and users' browser. In practise, developers tend to rely on third party software packages to construct their web infrastructure. Most of the time, user will trust the software package based on its reputation either it has became popular or it is developed by leading company. However, we may have no idea how it was implementation or even how it had be tested. In spite of the web infrastructure component could be based on a proven secure approach, the implementation might not be exactly the same as the approach suggested. Once a bug is found while it became widely used, the consequences could be disastrous. One of the good example is described in the following section.



\section{Objectives}
% General open questions. 
% Selection of particular question for study.
% s2n is memory safe.
The main objective of this project is to improve the web security by securing the critical network infrastructure component. Therefore, we propose this project, Securing the Web, in order to investigate the memory safety of the Amazon's s2n TLS implementation by using the Bounded Model Checking technique. As a result, a detailed experience report will be delivered and demonstrate how CBMC can be made use of in finding bugs in a heavily-tested critical software and showing the memory safety of the implementation.

% Therefore, we propose a project, Haskell Higher-Order Model Checking (HHOMC) to translate a Haskell program via Pattern-Matching Recursion Scheme (PMRS) into a Higher-Order Recursion Scheme (HORS) [24], in order to be model checked by Preface [26]. Meanwhile, a refinement process is implemented to complete the CEGAR loop and support Algebraic Data Types (ADT).
% All in all, HHOMC will be the first Haskell program verification software based on the model checking approach.


% which is an light-weight implementation of the TLS protocol, by using a bounded model checking tool, CBMC. Therefore, we propose this project to take a deep look into the implementation and examine the memory safety of the s2n implementation. The verification result will be documented in the project report.

% In this project, we are interested in the software security of web infrastructure components. 

% Therefore this project would be an experimental report to demonstrate how to make use of CBMC to verify the s2n implementation  memory safety of an industry code based.


\section{Main Contributions}
There are seven main contributions we have achieved in this project:
\begin{enumerate}
    \item Two hidden arithmetic overflow bugs are found in the s2n.
    \item Show CBMC is capable of finding hidden bugs in a heavily tested software package.
    \item Show the memory safety of the s2n implementation covered in our verification.
    \item Present an verification approach suitable to object-oriented software.
    \item Demonstrate two different ways to scale up the verification by using over-approximation. 
    \item Analyse the experiment results and limitations of the current approach.
    \item Document a detailed experimental report on the s2n verification using CBMC.
\end{enumerate}

% Main contributions in this project include:
% \begin{enumerate}
%     \item 
% \end{enumerate}

% The research questions we would ask as follows:

% \begin{enumerate}
%     \item Is the implementation memory safe?
%     \item Does the component based on any standard approach, such as encryption, protocol?
%     \item If the component based on some standard approaches, has it been implemented correctly?
%     \item Is there any undiscovered bugs?
%     \item Has the component been code reviewed or security audited?
% \end{enumerate}

% The main contribution of this paper is that this case study can serve as a reference to promote field engineers to adopt automated software analysis techniques. We have reported the detailed steps of applying CBMC to real-world embedded software busybox ls (how we obtained requirement prop- erties, how we set loop bounds, etc.) and concretely describing the benefit (i.e., high bug detection capability (Section V-C)) and the limitation of the technique (i.e., the manual cost for loop bound analysis (Section V-A) and the scalability problem for loop unwinding (Section V-C)) in practice. Thus, field engineers can estimate necessary efforts and benefits of applying CBMC from this case study and utilize the case study as a reference to apply automated software analysis techniques to their own projects.

\section{Dissertation Structure}
This dissertation consists of eight chapters that fully described the background knowledge and demonstrated all the work in this project. It is organised as follows: 
Chapter 1 is a brief introduction of this dissertation including the project motivation, objectives and the dissertation organisation.

Chapter 2 describes an overview of the existing software verification techniques, and their strengths and weakness as well as highlights the trade-offs in a software verification. Moreover, it also describes some researched approaches on using the bounded model checking technique.

Chapter 3 gives an overview of the Amazon's s2n TLS implementation and demonstrates the importance of ensuring the safety and correctness of such an implementation. Furthermore, it also comprehensively explains the fundamentals of bounded model checking and provides a thorough introduction of the bounded model checker, CBMC, being used in our verification.  

Chapter 4 describes the hypothesis of this dissertation and presents our verification approach in detail. 

Chapter 5 shows our experiment setup and the instruction for installing CBMC and s2n. Moreover, it describes the procedure of how a CBMC execution command is constructed.

Chapter 6 displays the overall verification results and explains each failure found during the verification. More importantly, it also shows the two arithmetic overflow bugs found in the s2n implementation and the corresponding solutions.
 
Chapter 7 discusses on how the memory safety of the s2n implementation can be shown, the trade-off we have made and the limitation of the verification.

Chapter 8 draws a conclusion for this project and discusses several possible future works. 