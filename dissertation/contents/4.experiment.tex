\chapter{Experiment Setup} \label{chpt:verification}

% In this chapter, we describe our verification in details 

% By the dependency analysis, we decide to verify the module of s2n in the following sequence (from the core part to the outer part)

% All the verification result are summarised in Section~\ref{sec:verificationResult}

% % ***************************** s2n_blob

% \section{s2n\_blob Verification}
% The header file of \code{s2n_blob} is shown as follows. 
% \begin{listing}[ht]
% \inputminted[frame=single, breaklines, linenos, numbersep=5pt, tabsize=4, firstline=20, fontsize=\footnotesize]{c}{./contents/code/s2n_blob.h} %firstline=359, lastline=385, 
% \caption{The header file - s2n\_blob.h}
% \end{listing}

% There are 2 functions are declared in the header file, however, we found there is no method call that uses the function \code{s2n_blob_init()}. Alternatively, the \code{struct s2n_blob} always be initialised directly (line 14). According to the usage of these functions, we constructed a simple harness, shown as follows:

% \begin{listing}[ht]
% \inputminted[frame=single, breaklines, linenos, numbersep=5pt, tabsize=4, firstline=8, fontsize=\footnotesize]{c}{./contents/code/harness/s2n_blob_harness.c} %firstline=359, lastline=385, 
% \caption{The harness for \textit{s2n\_blob.h}}
% \end{listing}

% First, we create a \code{data} array with non-deterministic size (line 13) in order to model all possible size of data being wrapped in \code{struct s2n_blob}. Then, we apply the function \code{s2n_blob_zero()} to the initialised \code{struct s2n_blob}. Since there is no loop involved in the above harness, no loop bound analysis is needed.

% % *************************** s2n_mem
% \section{s2n\_mem Verification}

% There are 4 functions are declared in \textit{s2n\_mem.h}, which aim at handling the memory allocation of \code{struct s2n_blob}. As their names suggest, \code{s2n_alloc()} allocates a \code{s2n_blob} with the given size, \code{s2n_realloc()} reallocates the given \code{s2n_blob} with the given size and \code{s2n_free()} is used to reclaim the memory of the given \code{s2n_blob}. \code{s2n_mem_init()} initialises those environment variables related to memory allocation, for example, the system page size. 

% \begin{listing}[ht]
% \inputminted[frame=single, breaklines, linenos, numbersep=5pt, tabsize=4, firstline=22, fontsize=\footnotesize]{c}{./contents/code/s2n_mem.h} %firstline=359, lastline=385, 
% \caption{The header file - s2n\_mem.h}
% \end{listing}

% After studying the commonly practise of these functions, we created an corresponding harness with respect to the usage contracts of these functions. First, we prepare the environment with an uninitialised variable \code{s2n_blob} and a non-deterministic value as the size of the \code{s2n_blob} (line 11-14). Then, we initialise those s2n environment variables by using \code{s2n_mem_init()} (line 16) and finally we call the functions in the following sequence, \code{s2n_alloc()}, \code{s2n_realloc()} and \code{s2n_free()}. Similar to the case of \code{s2n_blob}, no loops is involved in these functions. Therefore, no loop bound analysis is needed as well.

% \begin{listing}[ht]
% \inputminted[frame=single, breaklines, linenos, numbersep=5pt, tabsize=4, firstline=9, fontsize=\footnotesize]{c}{./contents/code/harness/s2n_mem_harness.c} %firstline=359, lastline=385, 
% \caption{The harness for \textit{s2n\_mem.h}}
% \end{listing}


% \section{s2n\_stuffer Verification}

% \subsection{s2n\_stuffer}
% \subsection{s2n\_stuffer\_text}
% \subsection{s2n\_stuffer\_pem}
% \subsection{s2n\_stuffer\_base64}

\section{Tool Setup and Configuration}
\label{sec:tsc}

\subsection{CBMC Installation}
Please follow the instructions to install CBMC in Mac OS X environment. For other environments, please refer to the instructions on the CBMC Github repository.
% \footnote{CBMC compilation instructions: https://github.com/diffblue/cbmc/blob/master/COMPILING}.
\begin{listing}[H]
\codeFromFile{bash}{./contents/resources/instruction/cbmc-installation.txt}{}{}
\caption{The instructions to install CBMC in Mac OS X environment.}
\end{listing}

\subsection{s2n Installation}
Please follow the instructions to install s2n in Mac OS X environment.

\begin{listing}[H]
\codeFromFile{bash}{./contents/resources/instruction/s2n-installation.txt}{}{}
\caption{The instructions to install s2n in Mac OS X environment.}
\end{listing}

\newpage
\section{Execution Environment}
\label{sec:ee}

The machine configuration for the experiment is shown below:
\begin{table}[H]
    \centering
    \begin{tabular}{c|c}
        \hline
        \hline
        Item & Value \\
        \hline
        \hline
        Processor & 2.5 GHz Intel Core i7\\
        Memory & 16 GB 1600 MHz DDR3\\
        Disk & 500GB Apple SSD SM0512G Media\\
        OS & OS X 10.11.5\\
        CBMC Version & 5.4\\
        CBMC Github Commit ID & d031ccc0f8ec82495310c3268066dcde5bc18c59\\
        s2n Github Commit ID & 42d9daf1813e60b9c8ad235096309486e9fbfa05\\
        OpenSSL & 1.0.2i-dev \\
        \hline
        \hline
    \end{tabular}
    \caption{Machine Configuration}
    \label{tab:mc}
\end{table}

\section{CBMC Execution Command Construction} \label{sec:cbmcc}
After setting up the environment and creating the verification harnesses, we have to construct the CBMC execution commands for each individual harness. We will demonstrate how a command can be constructed by going through one of the cases (\code{s2n_hmac_harness.c}) in detail:\\

First, we can run CBMC on the prepared harness directly, and it will report any missing header files that are required as shown in the follows:
\begin{listing}[htp]
\codeFromFile{bash}{./contents/resources/instruction/step1.txt}{}{}
\caption{Extracting the missing header file}
\label{lbl:step1}
\end{listing}

By repeating the above step, we can extract all the required header files and then include each of them into the execution command by using the \code{-I} option. Next, we can use the help of the \code{grep} command to capture any missing function body as shown below:

\begin{listing}[htp]
\codeFromFile{bash}{./contents/resources/instruction/step2.txt}{}{}
\caption{Extracting the missing function bodies}
\end{listing}

Since CBMC will display the warning messages in the \code{stderr} channel, therefore we have to redirect the warning messages into the \code{stdout} channel in order to capture them by using the \code{grep} command. Moreover, we also use \code{--unwind 1} to limit the number of iterations as one so that it can also cover those functions called in a loop. The above steps are repeated until all the required source files are included in the execution command. Then, we can extract all the loops involved in this execution with their loop IDs given by CBMC as shown in the following:

\begin{listing}[htp]
\codeFromFile{bash}{./contents/resources/instruction/step3.txt}{}{}
\caption{Extracting all the loops involved the execution}
\end{listing}

With the above information, we can perform loop bounds analysis for all the involved loops as explained in Section~\ref{sec:lba}. After all the LUBs have been determined, each of them can be specified in the execution command by using the \code{--unwindset} option. Then, we can test the termination of the constructed execution command as shown as follows:

\begin{listing}[H]
\codeFromFile{bash}{./contents/resources/instruction/step4.txt}{}{}
\caption{Testing the termination of the constructed execution command}
\end{listing}

The above verification result shows that the CBMC execution terminates normally, which means all involved loops are well bounded by the given LUBs. Moreover, the absence of warning messages means all the required files, both header files and source files, are included in the execution. Hence, the CBMC execution command for \code{s2n_hmac_harness.c} has been successfully constructed.

\section{CBMC Parameter Setting} \label{sec:cbmcps}
After the CBMC execution commands have been constructed, we can perform the verification by enabling those built-in checkers in CBMC according to our verification interest. Since we are interested in checking all the memory safety properties that are supported by CBMC as listed in Section~\ref{sec:hypothesis}. We will simply enable all the built-in checkers for our verification. In order to prevent an indefinite verification, set 5 hours as the timeout for the experiments. However, CBMC does not support timeout feature. Hence, we have created a helper tool to enable all the built-in checkers on verification, terminate the process on timeout and export the verification result into an output file. The usage and the implementation of the helper tool are listed below:


\begin{listing}[ht]
\codeFromFile{bash}{./contents/code/helper-tools/verifier-usage.txt}{The usage of verifier.sh}{}

\codeFromFile{bash}{./contents/code/helper-tools/verifier.sh}{helper-tools/verifier.sh}{}
\caption{The script usage and the implementation of the helper tool for assisting our verification}
\label{listing:3}
\end{listing}




