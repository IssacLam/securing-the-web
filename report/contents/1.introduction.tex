% What is the problem?
% Why is it interesting and important?
% Why is it hard? (E.g., why do naive approaches fail?)
% Why hasn't it been solved before? (Or, what's wrong with previous proposed solutions? How does mine differ?)
% What are the key components of my approach and results? Also include any specific limitations.

\chapter{Introduction}

\section{Motivation} \label{sec:heartbleed bug}
%% add affect & how serious it is %%
Failure of ensuring the safety of web infrastructure components could cause confidential information leakage and fatal damages. In April 2014, a simple careless mistake causes a security nightmare for the Internet, the Heartbleed Bug\footnote{The Heartbleed Bug: http://heartbleed.com/}. It is a serious vulnerability in the OpenSSL cryptographic library\footnote{The OpenSSL github repository: https://github.com/openssl/openssl}, an extensively used Transport Layer Security (TLS) protocol implementation, that is disclosed by Google's Security team and Codenomicon separately. This vulnerability allows attacker to steal sensitive information from HTTPS services, and even impersonate services and users. About half a million (17.5\%) of trusted HTTPS websites was influenced, including banking services, email services, cloud storage, and any web services using the vulnerable versions of OpenSSL, according to the Secure Sockets Layer (SSL)\footnote{Secure Sockets Layer: the predecessor of TLS, often be called TLS/SSL} survey conducted by the Netcraft \cite{9_heartbleed}.

%% Explain how it fails and can be avoid by using software verification %%
This security weakness occurred in the TLS "heartbeat" extension, which is a keep-alive feature intended to ensure the connection by sending an arbitrary payload to and exacting to receive the exact same copy from the other end. However, the implementation wrongly trust the non-sanitised user input and simply use it as the length parameter of \code{memcpy()} without any bounds check \cite{6_heartbleed_bug}. This missing of a proper input validation makes the reading of the payload beyond the end of the buffer. Hence, attackers are able to extract the memory contents, protected by the vulnerable versions of OpenSSL in the server, including identifications of service providers, secret keys for encrypting communications and users login information. In other words, they can steal data from communications directly and impersonate the services and users. However, such a serious security vulnerability caused by buffer overruns can be easily detected using formal software verification techniques, for example, Bounded Model Checking, which is demonstrated by the study of M. Voelter et al. \cite{Voelter:2015:TIS:2846696.2846698}. Therefore, not only the approach itself must be formally proven secure, but also the corresponding implementations are necessary to be verified.

Consider that the data integrity and privacy of Internet communications for the online applications and virtual private networks (VPNs) are provided by TLS/SSL. Therefore, ensuring the safety of the implementation of TLS/SSL is particular importance to securing the web. This project is motivated by the necessity of insuring such critical software components and the success of software verification, hence, we aim at verifying memory safety of the Amazon's s2n, which is an light-weight implementation of the TLS protocol, by using a bounded model checking tool, CMBC.


% \begin{figure}
%     \centering
%     \begin{minted}[fontsize=\footnotesize, linenos, frame=single]{c}
%     struct {
%         uint16 payload_length;
%         opaque payload[HeartbeatMessage.payload_length];
%         ...
%     } HearbeatMessage;
%     \end{minted}
%     \caption{The Heartbeat vulnerable code.}
%     \label{fig:thvc}
% \end{figure}



% Towards Improving Software Security using Language Engineering and mbeddr C.
% Many of the software security weakness originate from careless or wrong use of programming language.
% Software security refer to the security properties of a software;s implementation.

% The study further conclude that most of the software security weakness caused by careless or wrong use of programming languages, such as the Heartbleed bug, are possible to be prevented under a formal software verification.


% The proof that the program satisfies the specification follows from the fact that all trajectories are in the abstraction and this abstraction does not intersect the bad states.


% Internet surfing has become one of the important parts of life for most of the human being. No matter what we use it for, such as reading the daily news, watching our favourite sports or shopping online, our personal or sensitive information is always involved in the data transmission. Moreover, we will also assume that those data are safely handled by the web application provider. However, there are many web infrastructure components, such as establishing the connection, data encryption and security involved in the data transmissions between the actual application and users' browser. In practise, developers tend to rely on third party software packages to construct their web infrastructure. Most of the time, user will trust the software package based on its reputation either it has became popular or it is developed by leading company. However, we may have no idea how it was implementation or even how it had be tested. In spite of the web infrastructure component could be based on a proven secure approach, the implementation might not be exactly the same as the approach suggested. Once a bug is found while it became widely used, the consequences could be disastrous. One of the good example is described in the following section.



\section{Objectives}
% General open questions. 
% Selection of particular question for study.
% s2n is memory safe.
The main objective for this project is to verify the memory safety of the Amazon's s2n, which is an light-weight implementation of the TLS protocol, by using a bounded model checking tool, CMBC.

% They act as the critical roles for ensuring the security of a web application, moreover, there is no software verification done on them available online. Hence, we are going to take a deep look into the implementation and verify the satisfiability of the above two software packages. The verification result will be documented in the project report.
% In this project, we are interested in the software security of web infrastructure components. 

The research questions we would ask as follows:

\begin{enumerate}
    \item Is the implementation memory safe?
    \item Does the component based on any standard approach, such as encryption, protocol?
    \item If the component based on some standard approaches, has it been implemented correctly?
    \item Is there any undiscovered bugs?
    \item Has the component been code reviewed or security audited?
\end{enumerate}


\section{Main Achievements}
Main achievements in this project include:

\section{Dissertation Structure}
This dissertation is composed of ?? chapters, fully describing the background theory and demonstrating all my work for this project.

Chapter 1 is ...