\chapter{Experimental Evaluation}
\section{Execution Environment}
% \subsection{Big / little endian}
% http://searchnetworking.techtarget.com/definition/big-endian-and-little-endian

% \section{Experiment Setup}

\section{Results}
\section{Repeatability}
% other people can validate
\section{Threat to validity} % write about 2 pages
% find some papers

% Andrews, J.H., Briand, L.C., Labiche, Y.: Is mutation an appropriate tool for testing experi-
% ments? In: International Conference on Software Engineering (2005)

% Threats to Validity
% • Internal:
% • The programs, test pools, and faults used as is
% • No guarantee that test pools have the same detection power and
% coverage
% • Except Space, others have “realistic” hand seeded faults
% • Programs could be of varying complexity
% • Other mutation operators could produce varying results
% • Fixed size of test suite
% • External:
% • Relates to our ability to generalize the results of the experiment to industrial practice
% • Only one program with real faults used

% • Construct:
% • Concerns the way we defined our measurement
% • Does it measure the detection power of test sets
% and detectability of faults?
% • This was justified before
% • Conclusion:
% • Relates to subject selection, data collection, validity
% of the statistical tests, and measurement reliability
% • Addressed during the design of the experiment
