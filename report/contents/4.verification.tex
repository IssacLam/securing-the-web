\chapter{Experiment} \label{chpt:verification}

In this chapter, we describe our verification in details 

By the dependency analysis, we decide to verify the module of s2n in the following sequence (from the core part to the outer part)

All the verification result are summarised in Section~\ref{sec:verificationResult}

% % ***************************** s2n_blob

% \section{s2n\_blob Verification}
% The header file of \code{s2n_blob} is shown as follows. 
% \begin{listing}[ht]
% \inputminted[frame=single, breaklines, linenos, numbersep=5pt, tabsize=4, firstline=20, fontsize=\footnotesize]{c}{./contents/code/s2n_blob.h} %firstline=359, lastline=385, 
% \caption{The header file - s2n\_blob.h}
% \end{listing}

% There are 2 functions are declared in the header file, however, we found there is no method call that uses the function \code{s2n_blob_init()}. Alternatively, the \code{struct s2n_blob} always be initialised directly (line 14). According to the usage of these functions, we constructed a simple harness, shown as follows:

% \begin{listing}[ht]
% \inputminted[frame=single, breaklines, linenos, numbersep=5pt, tabsize=4, firstline=8, fontsize=\footnotesize]{c}{./contents/code/harness/s2n_blob_harness.c} %firstline=359, lastline=385, 
% \caption{The harness for \textit{s2n\_blob.h}}
% \end{listing}

% First, we create a \code{data} array with non-deterministic size (line 13) in order to model all possible size of data being wrapped in \code{struct s2n_blob}. Then, we apply the function \code{s2n_blob_zero()} to the initialised \code{struct s2n_blob}. Since there is no loop involved in the above harness, no loop bound analysis is needed.

% % *************************** s2n_mem
% \section{s2n\_mem Verification}

% There are 4 functions are declared in \textit{s2n\_mem.h}, which aim at handling the memory allocation of \code{struct s2n_blob}. As their names suggest, \code{s2n_alloc()} allocates a \code{s2n_blob} with the given size, \code{s2n_realloc()} reallocates the given \code{s2n_blob} with the given size and \code{s2n_free()} is used to reclaim the memory of the given \code{s2n_blob}. \code{s2n_mem_init()} initialises those environment variables related to memory allocation, for example, the system page size. 

% \begin{listing}[ht]
% \inputminted[frame=single, breaklines, linenos, numbersep=5pt, tabsize=4, firstline=22, fontsize=\footnotesize]{c}{./contents/code/s2n_mem.h} %firstline=359, lastline=385, 
% \caption{The header file - s2n\_mem.h}
% \end{listing}

% After studying the commonly practise of these functions, we created an corresponding harness with respect to the usage contracts of these functions. First, we prepare the environment with an uninitialised variable \code{s2n_blob} and a non-deterministic value as the size of the \code{s2n_blob} (line 11-14). Then, we initialise those s2n environment variables by using \code{s2n_mem_init()} (line 16) and finally we call the functions in the following sequence, \code{s2n_alloc()}, \code{s2n_realloc()} and \code{s2n_free()}. Similar to the case of \code{s2n_blob}, no loops is involved in these functions. Therefore, no loop bound analysis is needed as well.

% \begin{listing}[ht]
% \inputminted[frame=single, breaklines, linenos, numbersep=5pt, tabsize=4, firstline=9, fontsize=\footnotesize]{c}{./contents/code/harness/s2n_mem_harness.c} %firstline=359, lastline=385, 
% \caption{The harness for \textit{s2n\_mem.h}}
% \end{listing}


% \section{s2n\_stuffer Verification}

% \subsection{s2n\_stuffer}
% \subsection{s2n\_stuffer\_text}
% \subsection{s2n\_stuffer\_pem}
% \subsection{s2n\_stuffer\_base64}

\section{Tool Setup and Configuration}
\label{sec:tsc}


\section{Execution Environment}
\label{sec:ee}
% \subsection{Big / little endian}
% http://searchnetworking.techtarget.com/definition/big-endian-and-little-endian

% \section{Experiment Setup}
The machine configuration for the experiment is shown below:
\begin{table}[H]
    \centering
    \begin{tabular}{c|c}
        \hline
        \hline
        Item & Value \\
        \hline
        \hline
        Processor & 2.5 GHz Intel Core i7\\
        Memory & 16 GB 1600 MHz DDR3\\
        Disk & 500GB Apple SSD SM0512G Media\\
        OS & OS X 10.11.5\\
        CBMC Version & 5.4\\
        CBMC Github Commit ID & d031ccc0f8ec82495310c3268066dcde5bc18c59\\
        s2n Github Commit ID & 42d9daf1813e60b9c8ad235096309486e9fbfa05\\
        OpenSSL & 1.0.2i-dev \\
        \hline
        \hline
    \end{tabular}
    \caption{Machine Configuration}
    \label{tab:mc}
\end{table}

\section{CBMC Parameter Setting} \label{sec:cbmcps}

\section{Over-Approximation on ANSI-C Library}

\section{A detailed show case for show the verification procedure.}