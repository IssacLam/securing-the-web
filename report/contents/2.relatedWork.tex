% Show wt CBMC / Bounded model checking can do

\section{Other approaches using BMC}
The use of bounded model checking techniques have been presented and widely researched over the last couple of decades. In this section, we discuss some experiences of applying model checking on software security audit and assurance. It is important to learn from the successful cases from others in order to observe how the model checking is being used effectively and what kind of contribution to did it make to the software development process and the overall quality.

% of model checking technique for automated formal verification of software has been presented and widely researched over the last couples of decades. In this section, we discuss some practical applications and experiences of applying model checking techniques on security audit and assurance. It is important to learn from the successful cases from others in order to observe how the model checking is being used effectively and what kind of effect it has contributed to the development process and the overall quality.

% \marginnote{Elaborate more about ASTREE?}


% % Not very helpful, may be talk about some scale up problem & comparison with CBMC
% *P. Cousot et al. \cite{CousotEtAl10-FMSDC}

% Traditional testing vs Abstract testing.
% Bridging the gap between test cases and requirements by abstract testing
\paragraph{Abstract testing} is a software testing approach proposed by F. Merz et al. \cite{Merz:2015:BGT:2837773.2837824}, which focuses on the relation between requirements and the corresponding test cases in order to replace the traditional test cases by an abstract one. Each abstract test cases is derived form the requirements and formulated into assumption and assertion statements on the source code level, which is often a one to one relation. Assumption and assertion statements are the constraints encoding the preconditions and postconditions respectively. Setting up these constraints for the testing environment with respect to non-deterministic values, rather using concrete values. Hence, an abstract test case is possible to represent a large number of concrete test cases, which can remarkably simplify the test cases generation and maintenance process. The experimental study showed that abstract testing is more capable and efficient to discover underlying bugs than traditional software testing. Moreover, the performance of CBMC is promising with less than 20s runtime on average for an abstract test case, which is suitable for agile-like development process. Once there is a change of requirement, it is easily and directly visible to the corresponding abstract test cases.

%  ********** may be related
% Unit Testing of Flash Memory Device Driver through a SAT-Based Model Checker
Besides the incomplete coverage of conventional testing stated above, M. Kim et al. \cite{4639323, Kim:2008:FVF:1429078.1429092, 5510242} also mentioned that once a violation is detected, it still require a large amount of human effort to locate where the violation occurred by replaying the scenario and tracing the execution step-by-step. Hence, model checking techniques is well-suited to overcoming these weaknesses of conventional testing method through exhaustive analyses. This study applied SAT-based bounded model checking technique to verify the functional correctness and increase the reliability of the device driver software for Samsung's OneNAND\textsuperscript{TM} flash memory. In order to extract the code level properties, the study also suggested a top-down approach to identify the properties from several design documents, such as software requirement specifications, architecture design specifications, and detailed design specifications. It demonstrated a practical way on how the functional correctness properties can be extracted from the real world environment. By replacing the conventional testing with constraint-based exhaustive testing, which is the same as abstract testing above, CBMC achieved a great success in the experiment that discovered some bugs including incomplete exceptions handling and logical bugs. This promising result shows that model checking techniques are mature enough to be applied on verifying an industrial level software.

% Unit Testing of Flash Memory Device Driver through a SAT-Based Model Checker
% One advantage of SAT-based model checking is that a target C code can be analysed directly without an abstract model, thereby enabling automated and bit-level accurate verification. 
% Software requirement specifications (SRS), Architecture design specifications (ADS), Detailed design specifications (DDS)

% Property-based Code Slicing for Efficient Verification of OSEK/VDX Operating Systems
% very useful paper and also about CBMC
\paragraph{Property-based code slicing}
Despite the fact that model checking is a powerful technique, it often requires more knowledge and cost than testing. In order to reduce the cost while maintain comprehensiveness, M. Park et al. \cite{DBLP:journals/corr/abs-1301-0042} proposed an approach that consists of three strategies: 1) property-based environment generation, 2) property-based abstraction, and 3) collaborative verification using model checking and testing. The study applied it to verify the Trampoline operating system. The result showed that the approach is able to reduce the verification cost by scaling down the target code, and to simplify the analysis process by localising the verification activity.

% working on it.

% property-based environment generation and model extraction techniques using static code analysis, which can be applied to both model checking and testing. Comparative experiments using random testing and model checking for the verification of assertions in the Trampoline kernel code show how our environment generation and abstraction approach can be utilised for efficient fault detection. 

% Testing is insufficient for identifying subtle issues due to its optimistic incompleteness.

% Model checking is a powerful technique that supports comprehensiveness, and is thus suitable for the verification of safety-critical systems. However, it generally requires more knowledge and cost more than testing.

\paragraph{False positive elimination}
Software verification based on abstract interpretation is scalable for verifying industrial level code bases, but imprecise. Many spurious errors are generated as the abstract interpretation over-approximated the execution traces than the program can actually perform. Manual investigation is needed to review each error, such manual efforts are costly to the development process. To overcome this problem, 
% Reducing false positive
% Reducing False Positives by Combining Abstract Interpretation and Bounded Model Checking
H. Post et al. \cite{4639322} and P. Darke et al. \cite{Kumar:2013:PRA:2491411.2494569} integrate model checking to reduce the number of false positive generated by the static analysis tools. The experimental result shows that 69\% of warnings have been successfully removed by CBMC. 

% Reducing False Positives by Combining Abstract Interpretation and Bounded Model Checking
% The study shows that CBMC can generally provide more counter-example than SATABS. Abstract interpretation uses an over-approximation on the set of possible execution trace. i.e., it considers more program execution paths than the program can actually perform. Introduce false positives.

% doi not found, false positive elimination
% Efficient Elimination of False Positives Using Bounded Model Checking
Another study by T. Muske et al \cite{tukaram2013efficient} points out the above approach could involve numerous verifications, and further proposes an approach consisting of three techniques to achieve a faster false positives elimination. By avoiding redundant equivalent assertions, results in a 60\% accelerated false positives elimination. 

% identify an assertion as being equivalent to an other assertion thus avoiding its verification. Then, tries to identify and skip a class of assertion verifications that will not eliminate a false positive.

% Software verification using abstract interpretation is scalable but imprecise. Model checking is precise in verifying a property but not scalable. Often, these two techniques are combined to achieve better precision.

% On the Formal Verification of Component-based Embedded Operating Systems
% system level verification - isolation of components for modular verification
\paragraph{Aspect-oriented programming technique} M. Ludwich and A. Frohlich \cite{Ludwich:2013:FVC:2433140.2433148} introduced an approach to formally verify both \textit{functional correctness} and \textit{safety properties} of embedded operating system components. Such components are supposed to be shifted between different domains, therefore, the implementation of formal verification must also be domain-independent. The study proposed a corresponding strategy that consists of three main steps: 1) creating contracts for each components that to be verified , 2) implementing such components with the corresponding contracts accordingly, and 3) performing software bounded model checking to verify whether the implementation of components respect to their contracts. The benefits of keeping the specification and implementation close to each other is that any violations in the implementation can be detected at the early stage of development. However, this may introduce an extra run-time overhead to its instances. Hence, those contracts should be partially eliminated from the implementations beyond the verification stage. The experiment made used of aspect-oriented programming technique to achieve this purpose, and further facilitate a modular verification by isolation different components. Note that components isolation is not suitable for verifying a monolithic design. More importantly, the experiment also demonstrated that the functional correctness and safety properties can be verified by the contracts captured from requirements and those properties generated by the model checker respectively. This approach is a kind of white box testing and is more suitable to carry out at the early stage or during the development.

% How Verified is My Code? Falsification-Driven Verification (T)
\paragraph{Falsification-Driven Verification} A. Groce et al. \cite{7372062} stated the idea of a "verification successful" in model checking or even theorem proving can be a sign of insufficient verification properties. In order to gain more knowledge of the meaning behind "successful" results, the study proposed a falsification-driven methodology that adapts mutation testing to show the weakness of current specification. Mutate both the harness and the program with a small syntactic change assuming a good test suit should be able to detect a bug introduced by such a change. Hence, the mutation kill rate can act as a measurement unit of the accuracy rate and correctness of a harness. As a better harness should able to detect more errors from the implementation. This approach is useful for ensuring the quality of harnesses, thus a more reliable verification. However, it requires many manual efforts to verify and examine the mutants while some of them can be semantically equivalent. It would be suitable for developers who are not familiar with formal verification but intend to verify a system.


% On the Formal Verification of Component-based Embedded Operating Systems
% Moreover, they demonstrate the approach causes no run-time overhead, since the adopted Software Model Checking techniques are deployed at compile-time. The functional correctness properties of a component are expressed by a contract containing assertions that represent class invariants, pre, and postconditions for each method declared in the component's interface. The safety properties are meant to be automatically generated by model checker.


% A contract is composed by the invariants of the class that defines the component and by a set of pre and postconditions. Class invariants are defined as a set of assertions that are always true for all instances of that class. The preconditions of a method define a set of assertions that must be true in case the method invocation terminated normally (ie. without triggering exceptions, which have not yet been addressed in our approach.) Pre and postcondition assertions can reason about the method's parameters, return value, and object state.

% Bridging the gap between test cases and requirements by abstract testing
% The havoc() procedure call in the example sets some global vari- ables to undefined (non-deterministic) values.

% \begin{itemize}
%     \item Software testing only consists of few selected cased for checking the correctness, which is sound but not complete.
%     \item Abstraction based verification techniques covers all possible input spaces, which is complete but may cause false positive.
%     \item Bounded Model Checking is sound, but only execution paths up to the specified length are checked; therefor not complete.
%     \item BMC with large enough bound is sound and complete.
% \end{itemize}


% no really helpful - configuration lifting
% Configuration Lifting: Verification meets Software Configuration
\paragraph{Configuration lifting} is a specification analysis technique presented by
H. Post and C. Sinz \cite{4639338}, which transforms all variants into a meta-program and facilitates the following specification analysis in three different domains: 1) inconsistencies within a feature model, 2) inconsistencies between the feature model and their implementations, and also 3) the coverage of run-time errors in all product variants. The study demonstrated the technique on verifying Linux kernel and device driver, which successfully found two bugs in the kernel configuration system. The experimental result showed that this technique is able to apply on a large size system included more than 4600 features and model checking on the meta-program provides a significant speed-up compared with traditional enumeration based analysis.

% Semiformal Verification of Embedded Software in Medical Devices Considering - Stringent Hardware Constraints
\paragraph{Hybrid approach} L. Cordeiro et al. \cite{5066674} propose a semi-formal verification approach combined dynamic and static verification to stress and cover the state spaces of embedded software exhaustively, in order to improve the coverage and reduces the verification time. This approach allows developers to reason both the functional and temporal properties quantitatively to guarantee the timeliness and correctness of the design. The experimental result shows that model checkers have  limitation to specify more complex temporal properties and state space explosion problem.


% % A Comparative Study of Software Model Checkers as Unit Testing Tools: An Industrial Case Study
% M. Kim et al. \cite{5510242} reported their experience in applying Blast and CBMC to testing the components of a storage platform software for flash memory.

% Significant additional efforts are required to create an abstract target model, which is not affordable for most industrial projects.

% \cite{5510242}
% There have been several studies in which an SMC was
% applied to various target systems such as automotive
% software [47], microcontroller programs [48], Linux device
% drivers [45], [46], file systems [52], network filters [13], a
% protocol stack [40], and server applications [41]. For a flash
% storage platform, which is our main target domain, one
% recent study [32] analyzes flash file systems as an effort of
% the mini-challenge [35]. However, the majority of verification
% research [36], [28], [11] on flash file systems focus on
% the specifications of the file system design and not on an
% actual instance of implementation.

% % not really helpful
% *T. Vortler et al. \cite{7195716} presented a verification framework for applications for the embedded system operating system Contiki, based on CBC, supporting interrupts.

% % not really helpful (something about partial reduction) - verifying sensor node using CBMC
% % D. Bucur and M. Kwiatkowska \cite{BK11}

% % not really helpful - but a bit about CBMC
% *B. Schlich and S. Kowalewski \cite{Schlich:2009:MCC:1569778.1569782} studied 14 tools and summarised the limitations for verifying embedded software.


% % Crowdsourcing Program Preconditions via a Classification Game
% % D. Fava et al. \cite{Fava:2016:CPP:2884781.2884865} used CBMC to generate the Invariant / check whether the precondition can cause the postcondition

% % % Seems not really helpful
% % I. Wenzel et al. \cite{0fbeb641779543c98fb1d6ff2180664c} 
% % only use CBMC for test data generation.

% % talked about memory safety
% *H. Post and W. K \cite{Post:2007:ISA:1770498.1770525} able to verify the memory safety of Linux drivers using CBMC.

% % Something about CBMC maybe useful
% *L. Cordeiro et al. \cite{Cordeiro:2009:SBM:1747491.1747515}
% CBMC has limitations due to the fact that the size of the propositional formulae increases significantly in the presence of large datapaths and high-level information is lost when the verification conditions are converted into propositional logic.


% \cite{Petrenko:2015:MTA:2744769.2747935}





% % To the best of our knowledge, there is no work that considers .... As a result, our main contribution is ....